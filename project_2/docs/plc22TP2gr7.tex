\documentclass[11pt,a4paper]{report}

\usepackage[utf8]{inputenc}
\usepackage{times}
\usepackage{multirow}
\usepackage{array}
\usepackage{graphicx}
\usepackage{url}
\usepackage{enumerate}
\usepackage{xspace}
\usepackage[portuguese,main=english]{babel}
\usepackage{cite}
\usepackage{listings}
\usepackage{titlesec}
\usepackage{color}
\parindent=0pt
\titleformat{\section}{\normalfont\Large\bfseries}{\S\thesection}{1em}{}
\parskip=2pt
\setlength{\oddsidemargin}{-1cm}
\setlength{\textwidth}{18cm}
\setlength{\headsep}{-1cm}
\setlength{\textheight}{23cm}
\lstset{
	basicstyle=\small, %o tamanho das fontes que são usadas para o código
	numbers=left, % onde colocar a numeração da linha
	numberstyle=\tiny, %o tamanho das fontes que são usadas para a numeração da linha
	numbersep=5pt, %distancia entre a numeração da linha e o codigo
	breaklines=true, %define quebra automática de linha
    frame=tB,  % caixa a volta do codigo
	mathescape=true, %habilita o modo matemático
	escapeinside={(*@}{@*)} % se escrever isto  aceita tudo o que esta dentro das marcas e nao altera
}
\lstdefinestyle{custompy}{
  belowcaptionskip=1\baselineskip,
  breaklines=true,
  frame=tB,
  xleftmargin=\parindent,
  language=Python,
  showstringspaces=false,
  basicstyle=\footnotesize\ttfamily,
  backgroundcolor=\color{white},
  keywordstyle=\bfseries\color{red},
  commentstyle=\itshape\color{blue},
  identifierstyle=\color{black},
  stringstyle=\color{green},
}

\title{Processamento de Linguagens e Compiladores (3º Ano LCC)\\ 
      \textbf{Project 2} \\ Project Report
      }
\date{\today}
\author{Bruno Dias da Gião\\ A96544 \and Maria Filipa Rodrigues \\ A97536}

\begin{document}
\maketitle

\selectlanguage{portuguese}
\begin{abstract}
Lorem Ipsum
\end{abstract}
\selectlanguage{english}
\begin{abstract}
Lorem Ipsum
\end{abstract}

\tableofcontents

\chapter{Report} \label{chap:report}
\section{Introduction} \label{intro} 
\subsection{The ``Not-Quite-C-Nor-B-But-More-Like-C'' language Compiler}
\subsubsection{Introduction to the Report}  

\subsubsection{Historical background of BCPL, B and C}

\subsubsection{Historical background of Lexx, Yacc and PLY}
\subsubsection{Importance of this project}


\subsubsection{Background of the Project}
\subsubsection{Expansions of the Project}
\section{Methodology} \label{methodology}
\subsection{Theoretical Background} \label{theory}
\subsection{Practical component} \label{practice}
\subsection{Testing the code}

\section{Conclusion} \label{conclusion}

\chapter{Appendix} \label{chap:code}
\section{Codes}


\bibliography{bib}{}
\bibliographystyle{plain}
\end{document}

