% Template for taking notes\writing down exercise solutions
% In the specific context of mathematical Curricular Units

% In order to draw Finite State Machines\Turing Machines,
% we make use of the tikz-automata library

\documentclass[13.7pt]{beamer}
\mode<presentation>
\usetheme{CambridgeUS}
\usepackage{xspace}
\usepackage[english]{babel}
\usepackage{amsmath}
\usepackage{amsthm}
\usepackage{amssymb}
\usepackage{tikz}
\usepackage{enumerate}
\usetikzlibrary{automata, positioning, arrows}
\tikzset{->,
    >=stealth',
    node distance=3cm,
    every state/.style={thick, fill=gray!10},
    initial text=$ $,
}
\usepackage{listings}
\usepackage{xcolor}

\definecolor{codegreen}{rgb}{0,0.6,0}
\definecolor{codegray}{rgb}{0.5,0.5,0.5}
\definecolor{codepurple}{rgb}{0.58,0,0.82}
\definecolor{backcolour}{rgb}{0.95,0.95,0.92}

\lstdefinestyle{mystyle}{backgroundcolor=\color{backcolour},
    commentstyle=\color{codegreen},
    keywordstyle=\color{magenta},
    numberstyle=\tiny\color{codegray},
    stringstyle=\color{codepurple},
    basicstyle=\ttfamily\footnotesize,
    breakatwhitespace=false,
    breaklines=true,
    captionpos=b,
    keepspaces=true,
    numbers=left,
    numbersep=5pt,
    showspaces=false,
    showstringspaces=false,
    showtabs=false,
    tabsize=2
}

\lstset{style=mystyle}

\renewcommand\qedsymbol{QED}


\title[CSV conversion to JSON]
    {\textbf{First Project Presentation}\\
    \textbf{CSV conversion to JSON} \\
    \textit{LCC 3rd Year}
}

\author[A96544 \and A96536]
   {Bruno Dias da Gião \and Maria Filipa Rodrigues \\
    A96544 \and A97536
}

\date{\today}


\usecolortheme{default}
%\useoutertheme{infolines}
\setbeamercovered{transparent=0}
\setbeamertemplate{headline}[default]
\setbeamercolor{mycolor}{fg=white,bg=structure!30}
\setbeamertemplate{navigation symbols}{}
\addtobeamertemplate{footline}
    {\leavevmode%
    \hbox{%
    \begin{beamercolorbox}
        [
        wd=\paperwidth,
        ht=2.75ex,
        dp=.5ex,
        right,
        rightskip=1em
        ]
        {mycolor}%
    \usebeamercolor[fg]{navigation symbols}
    \insertslidenavigationsymbol%
    \insertframenavigationsymbol%
    \insertsubsectionnavigationsymbol%
    \insertsectionnavigationsymbol%
    \insertdocnavigationsymbol%
    \insertbackfindforwardnavigationsymbol%
    \end{beamercolorbox}%
    }%
    \vskip0.5pt
 }{}




\begin{document}


% Title Page
\frame{\titlepage}


% Table of Contents
\begin{frame}{Contents}
\tableofcontents
\end{frame}


\section{Introduction}

\subsection{Objectives}
% SLIDE 0
\begin{frame} {Objective of this Project}
    \begin{enumerate}
        \item{``De-Obfuscating' tabular data files into readable plain text}
    \pause\item{Testing and consolidating theoretical
            and practical knowledge of \textbf{Regular Expressions}}
    \end{enumerate}
\end{frame}
% SLIDE 1
\begin{frame} {Objective of this Presentation}
    \begin{enumerate}
        \item{Facilitating the reading and comprehension of both the project's
            report and its code.}
    \end{enumerate}
\end{frame}

\subsection{Historical Context}
% SLIDE 2
\begin{frame} {Historical Context}
    \begin{enumerate}
        \item{Origin of CSV}
  \pause\item{Origin of XLS}
  \pause\item{Origin of JavaScript and JSON}
    \end{enumerate}
\end{frame}

\subsection{Background}
% SLIDE 3
\begin{frame}{Motivation and Background}
    \begin{enumerate}
        \item{What was asked for the project?}
  \pause\item{Why is this a pertinent subject?}
    \end{enumerate}
\end{frame}

\subsection{Expanding this project}
% SLIDE 4
\begin{frame}{Expansion of the Project}
    \begin{enumerate}
        \item{Preprocessing XLS files}
  \pause\item{Using the JSON file in real contexts}
    \end{enumerate}
\end{frame}


\section{Practical Component}
\subsection{Methodology}
\begin{frame}{Methodology}
    \begin{enumerate}
        \item{Design Philosophy}
  \pause\item{Conception}
  \pause\item{Result}
    \end{enumerate}
\end{frame}
\subsection{Analysis}
\begin{frame}{Analysis}
    \begin{enumerate}
        \item{Regular Expressions}
  \pause\item{Python's re module}
  \pause\item{Text Filters}
    \end{enumerate}
\end{frame}
\subsection{Results}
\begin{frame}{Results}
    \begin{enumerate}
        \item{Examples}
    \end{enumerate}
\end{frame}
\subsection{XL pre-processing}
\begin{frame}{XL preprocessing}
    \begin{enumerate}
        \item{How are XL files stores}
  \pause\item{How it can be done}
  \pause\item{Why it wasn't done}
    \end{enumerate}

\end{frame}

\section{Conclusion}
\begin{frame}{Final Remarks}
    \begin{enumerate}
        \item{Our opinions on our work}
  \pause\item{Future work}
  \pause\begin{enumerate}
      \item{Real world applications?}
        \end{enumerate}
    \end{enumerate}
\end{frame}
\begin{frame}
    \huge{\textbf{Thank You!}}
\end{frame}
\end{document}
